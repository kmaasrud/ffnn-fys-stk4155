\documentclass[../main.tex]{subfiles}

\begin{document}
\section{Theory}
\subsection{Logistic regression}
Logistic regression is a method for classifying a set of input variables \ensuremath{\boldsymbol{x}} to an output or class \ensuremath{y_i, i=1,2, \ldots,K} where $K$ is the number of classes. The review in this section is based on Hastie et al. \cite[ch.~4]{HastieTrevor2009EoSL},  and the reader is referred to this book for a more detailed explanation of topic. The prediction of output classes which the input variables belongs to is based on the design matrix \ensuremath{\boldsymbol{X}\in\mathbb{R}^{n\times p}} that contains $n$ samples that each carry $p$ features. We distinct between \textit{hard} and \textit{soft classification}, which determines the input variable to a class deterministically or the probability that a given variable belongs in a certain class. The logistic regression model is given on the form
 
\begin{align}
    \begin{split}
        \log\frac{p(G=1|X=x)}{p(C=K|X=x)}&=\beta_{10}+\beta_1^Tx \\
        \log\frac{p(G=2|X=x)}{p(C=K|X=x)}&=\beta_{20}+\beta_2^Tx \\
        &\vdots\\ 
        \log\frac{p(G=K-1|X=x)}{p(C=K|X=x)}&=\beta_{(K-1)0}+\beta_{K-1}^Tx.
    \end{split}
\end{align}

We consider the binary, two-class case with \ensuremath{y_i \in [0,1]}. The probability that a given input variable $x_i$ belongs in class $y_i$ is given by the Sigmoid-function (also called logistic function):
\begin{align}
    p(x) = \frac{e^x}{1+e^x}=\frac{1}{1+e^{-x}}.
\end{align}

A set of predictors, \ensuremath{\boldsymbol{\beta}}, which we want to estimate with our data then gives the probabilities 

\begin{align}
    p(y_i=1|x_i,\boldsymbol{\beta})=\frac{\exp\left(\boldsymbol{\beta}^Tx_i\right)}{1+\exp\left(\boldsymbol{\beta}^Tx_i\right)} \\
    p(y_i=0|x_i,\boldsymbol{\beta})=1-p(y_i=1|x_i,\boldsymbol{\beta}).
\end{align} We define the set of all possible outputs in our data set \ensuremath{\mathcal{D}(x_i,y_i)}. Further, we assume that all samples \ensuremath{\mathcal{D}(x_i,y_i)} are independent and identically distributed. Now we can approximate the total likelihood for all possible outputs of \ensuremath{\mathcal{D}} by the product of the individual probabilities \cite[p.~120]{HastieTrevor2009EoSL} of a specific output $y_i$:

\begin{align}
    P(\mathcal{D}|\boldsymbol{\beta}) = \prod_{i=1}^n[p(y_i=1|x_i\boldsymbol{\beta})]^{y_i}[1-p(y_i=1|x_i,\boldsymbol{\beta})]^{1-y_i}.
    \label{eq:likelihood}
\end{align} We want to maximize this probability by using the maximum likelihood estimator (MLE). By taking the logarithm of \cref{eq:likelihood}, we obtain the log-likelihood in \ensuremath{\boldsymbol{\beta}}

\begin{align}
    \log P(\mathcal{D}|\boldsymbol{\beta}) = \sum_{i=1}^n[y_i\log p(y_i=1|x_i\boldsymbol{\beta})+(1-y_i)\log(1-p(y_i=1|x_i,\boldsymbol{\beta}))].
    \label{eq:log-likelihood}
\end{align}

By reordering the logarithms and taking the negative of \cref{eq:log-likelihood}, we obtain the \textit{cross entropy}

\begin{align}
    \mathcal{C}(\boldsymbol{\beta})=-\sum_{i=1}^n\left[y_i\boldsymbol{\beta}^Tx_i-\log\left(1+\exp\left(\boldsymbol{\beta}^Tx_i\right)\right)\right].
    \label{eq:cross-entropy}
\end{align} The cross entropy is used as our cost function for logistic regression. We minimize the cross entropy, which is the same as maximizing the log-likelihood, and obtain

\begin{align}
    \frac{\partial\mathcal{C}(\boldsymbol{\beta})}{\partial\boldsymbol{\beta}}=-\sum_{i=1}^nx_i(y_i-p(y_i=1|x_i,\boldsymbol{\beta}))=0.
\end{align} The second derivative of this quantity is

\begin{align}
    \frac{\partial^2\mathcal{C}(\boldsymbol{\beta})}{\partial\boldsymbol{\beta}\partial\boldsymbol{\beta}^T}=\sum_{i=1}^nx_ix_i^Tp(y_i=1|x_i,\boldsymbol{\beta})(1-p(y_i=1|x_i,\boldsymbol{\beta})).
\end{align} These expressions can be written more compactly by defining the diagonal matrix \ensuremath{\boldsymbol{W}} with elements \ensuremath{p(y_i=1|x_i,\boldsymbol{\beta})(1-p(y_i=1|x_i,\boldsymbol{\beta}))}, \ensuremath{\boldsymbol{y}} as the vector with our $y_i$s values and \ensuremath{\boldsymbol{p}} as the vector of fitted probabilities. We can then express the first and second derivatives in matrix form

\begin{align}
    \frac{\partial\mathcal{C}(\boldsymbol{\beta})}{\partial\boldsymbol{\beta}}&=-\boldsymbol{X}^T(\boldsymbol{y}-\boldsymbol{p}) \\
    \frac{\partial^2\mathcal{C}(\boldsymbol{\beta})}{\partial\boldsymbol{\beta}\partial\boldsymbol{\beta}^T} &= \boldsymbol{X}^T\boldsymbol{W}\boldsymbol{X},
\end{align} also known as the Jacobian and Hessian matrices, respectively. We will use the stochastic gradient descent (SGD) (\cref{sec:sgd}) to find the optimal parameter \ensuremath{\boldsymbol{\beta}}. 

\subsection{Datasets}
We will analyze the following datasets:
\subsubsection{MNIST}
The famous MNIST dataset is a collection of handwritten numbers, as $28\times 28$ grayscale images. It comes in two sets, a training set with $60,000$ images, and a testing set with $10,000$ images. In this report, we will model the inputs as a $28\times 28 = 784$-dimensional vector, and the output as a $10$-dimensional state vector, with each dimension representing the corresponding digit.

\subsection{Stochastic Gradient Descent}\label{sec:sgd}
\textit{Gradient descent} describes the process of finding a local minimum of a function (the cost function, in our case) by following the negative value of the gradient at each point, stepwise. \textit{Stochastic gradient descent} or SDG is a way of increasing the numerical efficiency of this process, by doing this process stochastically.

This involves randomly dividing the training data into a given number of \textit{mini batches}. For each mini batch, the gradient is found by averaging the gradient value each mini batch sample has. Then the weights and biases are updated (take a step down the "slope") and the process is repeated for the rest of the mini batches. The updating done at each mini batch is expressed mathematically as

\begin{align*}
    w&\rightarrow w' = w - \frac{\eta}{m}\sum_i^m \nabla C_{i,w} \\
    b&\rightarrow b' = b - \frac{\eta}{m}\sum_i^m \nabla C_{i,b},
\end{align*}

where $m$ is the number of datapoints in the mini batches and $\nabla C_i$ is the gradient at each individual data point. After exhausting all the training data, we have finished a so-called \textit{epoch}, of which we can perform as many as necessary.

\end{document}